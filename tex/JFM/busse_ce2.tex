% This is file JFM2esam.tex
% first release v1.0, 20th October 1996
%       release v1.01, 29th October 1996
%       release v1.1, 25th June 1997
%       release v2.0, 27th July 2004
%       release v3.0, 16th July 2014
%   (based on JFMsampl.tex v1.3 for LaTeX2.09)
% Copyright (C) 1996, 1997, 2014 Cambridge University Press

\documentclass{jfm}
\usepackage{graphicx}
%\usepackage{epstopdf, epsfig}
\usepackage{physics}

\usepackage{enumitem,amssymb}
\newlist{todolist}{itemize}{2}
\setlist[todolist]{label=$\square$}
\usepackage{pifont}
\newcommand{\cmark}{\ding{51}}%
\newcommand{\xmark}{\ding{55}}%
\newcommand{\done}{\rlap{$\square$}{\raisebox{2pt}{\large\hspace{1pt}\cmark}}%
\hspace{-2.5pt}}
\newcommand{\wontfix}{\rlap{$\square$}{\large\hspace{1pt}\xmark}}

\newtheorem{lemma}{Lemma}
\newtheorem{corollary}{Corollary}

\newcommand{\cz}{c_{\zeta}}
\newcommand{\cs}{c_{\psi}}
\newcommand{\ct}{c_{\theta}}
\newcommand{\csz}{c_{\psi \zeta}}
\newcommand{\czs}{c_{\zeta \psi}}
\newcommand{\czz}{c_{\zeta \zeta}}
\newcommand{\ctz}{c_{\theta \zeta}}
\newcommand{\czt}{c_{\zeta \theta}}
\newcommand{\cst}{c_{\psi \theta}}
\newcommand{\cts}{c_{\theta \psi}}
\newcommand{\Ray}{\mbox{\textit{Ra}}}  % Rayleigh number


\shorttitle{CE2 Busse Annulus}
\shortauthor{J. S. Oishi, S. M. Tobias, K. J. Burns, J. B. Marston}

\title{Direct Statistical Simulation of the Busse Annulus}

\author{Jeffrey S. Oishi\aff{1}
  \corresp{\email{joishi@bates.edu}},
  Steven M. Tobias\aff{2},
  Keaton J. Burns\aff{3
}
 \and J. B. Marston\aff{4}}

\affiliation{\aff{1}Department of Physics \& Astronomy, Bates College,
Lewiston, ME 04240, USA
\aff{2}Department of Applied Mathematics, University of
Leeds, Leeds LS2 9JT, UK
\aff{3}Center for Computational Astrophysics, Flatiron Institute, New York, NY 10010, USA
\aff{4}Department of Physics, Brown University, Providence, RI 02912, USA
}

\begin{document}

\maketitle

\begin{abstract}
Turbulent systems generate and interact with mean flows in a wide variety of natural systems.
Understanding the nature of these interactions, particularly for systems far from equilibrium, remains a paramount priority in understanding large-scale flows on planets and stars.
The fundamental problem in studying such systems via direct numerical simulation (DNS) is the fact that the smallest scales can have a significant impact on the mean flows, even when they are very widely separated.
One way to make progress is to study the statistics of the flow rather than detailed flow variables themselves.
By expanding around the mean flow in terms of equal-time cumulants, we can arrive at a closed set of equations of motion for the cumulants.
Here, we present results using an expansion terminated at the second cumulant (CE2) for rapidly rotating thermal convection in an annulus.
CE2 discards eddy-eddy interactions that yield eddies; it is fundamentally quasi-linear.
We focus on a particular case in which the direct numerical simulation yields an initial three-jet solution that is unstable to a two-jet solution.
Interestingly, CE2 predicts a stable three-jet solution, though we find that by biasing the initial conditions to favor certain symmetries, CE2 reproduces the DNS results.
\end{abstract}

\begin{keywords}
\end{keywords}

\section{Introduction}
\label{sec:intro}


Rationale for statistical methods
What has been done before in DSS
Model problems typically stochastically driven; this is not

\section{Model Equations}
\label{sec:model-eqations}

We consider the equations for the Busse annulus.
We write the equation of motion in terms of the $z$-component of the vorticity, $\zeta$.

\begin{equation}
  \label{eq:zeta_eom}
  \pdv{\zeta}{t} + J(\psi, \zeta) - \beta \pdv{\psi}{x} = -\frac{\Ray}{\Pran} \pdv{\theta}{x} -C |\beta|^{1/2} \zeta + \laplacian{\zeta}
\end{equation}
The streamfunction $\psi$ is defined by
\begin{equation}
  \label{eq:zeta_def}
  \zeta = \laplacian{\psi},
\end{equation}
which means the $x$ velocity $u = -\pdv*{\psi}{y}$ and the $y$ velocity $v = \pdv*{\psi}{x}$. $J(A, B) = \pdv*{A}{x}\pdv*{B}{y} - \pdv*{A}{y}\pdv*{B}{x}$ is the Jacobian. 
The perturbed temperature is given by $\theta$, which obeys the dynamical equation
%
\begin{equation}
  \label{eq:theta}
  \pdv{\theta}{t} + J(\psi, \theta) = -\pdv{\psi}{x} + \frac{1}{\Pran} \nabla^2 \theta.
\end{equation}
The system is thus governed by four dimensionless parameters, $\beta$, $C$, $\Pran = \nu/
\kappa$, and $\Ray = \alpha g \Delta T d^3/\nu \kappa$

For the CE2 expansion, we need equations for the first cumulants $\cz$ and $\ct$, and the second cumulants $\czs$, $\czz$, $\ctz$, and $\czt$. 
While second cumulants involving $\psi$ and $\zeta$ are related by the gradient operators, e.g. $\czz = \nabla_2^2 \czs$, those involving $\theta$ require use of the symmetry $\ctz(y_1, y_2, \xi) = \czt(y_2, y_1, -\xi)$.

We solve dynamical equations for $\cz$,
\begin{equation}
  \label{eq:cz}
  \pdv{\cz}{t} = - \qty(\pdv{y_1} + \pdv{y_2}) \pdv{\csz}{\xi}\eval_{\xi \to 0}^{y_1 \to y_2} - C |\beta|^{1/2} \cz + \pdv[2]{\cz}{y_1},
\end{equation}

\begin{equation}
  \label{eq:czz}
  \begin{split}
    \pdv{\czz}{t} &= \pdv{\cs}{y_1} \pdv{\czz}{\xi} - \qty(\pdv{\cz}{y_1} - \beta) \pdv{\csz}{\xi} - \pdv{\cs}{y_2} \pdv{\czz}{\xi}  + \qty(\pdv{\cz}{y_2} - \beta) \pdv{\czs}{\xi}\\
    &+ \frac{\Ray}{\Pran} \qty(\pdv{\czt}{\xi} -  \pdv{\ctz}{\xi}) - 2C |\beta|^{1/2} \czz + (\nabla_1^2 + \nabla_2^2) \czz    
  \end{split}
\end{equation}

The first cumulant for $\theta$ evolves according to
\begin{equation}
  \label{eq:ct}
  \pdv{\ct}{t} = - \qty(\pdv{y_1} + \pdv{y_2}) \pdv{\cst}{\xi} \eval_{\xi \to 0}^{y_1 \to y_2} + \frac{1}{\Pran} \pdv[2]{\ct}{y_1}.
\end{equation}
For the second cumulant in $\theta$, we evolve $\ctz$,
\begin{equation}
  \label{eq:ctz}
  \begin{split}
    \pdv{\ctz}{t} &= \qty(\pdv{\cs}{y_1} - \pdv{\cs}{y_2}) \pdv{\ctz}{\xi} - \qty(1 + \pdv{\ct}{y_1}) \pdv{\csz}{\xi} + \qty(\pdv{\cz}{y_2} - \beta) \pdv{\cts}{\xi}\\
    &  - C |\beta|^{1/2} \ctz + \frac{1}{\Pran}\qty(\nabla_1^2 + \nabla_2^2) \ctz + \cdots,    
  \end{split}
\end{equation}
where $\cdots$ indicates the terms needed to ensure symmetry under exchange of $x_1, y_1$ and $x_2, y_2$.

\subsection{Parity}
\label{sec:parity}

We solve the system subject to impenetrable, stress-free boundary conditions in the $y$ dimension; it is periodic in $x$. 
This means that $\theta$, $\zeta$, and $\psi$ all have odd parity.
The action of the zonal average preserves the parity, and so we discretize the first cumulants $\cz$, $\ct$ using a $\sin$ series. 

\subsection{Numerical Methods}
\label{sec:numerical}

We use \emph{Dedalus} to solve both the direct equations (\ref{eq:zeta_eom} - \ref{eq:theta}) and the CE2 model equations, (\ref{eq:cz})-~(\ref{eq:ctz}).


\section{Results}
\label{sec:results}

In order to facilitate comparison with previous work, we adapt the same run naming scheme as in \citet{2018RSPSA.47480422T}.

\subsection{Hovoller}
\label{sec:hov}

time averaged 1st and 2nd cumulants

\subsection{3 Jet to 2 Jet Transitions}
\label{sec:3->2}

In run A, a curious effect occurs: the CE2 results return a 3-jet solution, while the DNS yields a 2-jet solution.
This is quite similar to the quasilinear run in \citet{2018RSPSA.47480422T}, which also produces a 3-jet solution.
Given that CE2 is fundamentally a quali-linear theory, this is unsurprising.
One might hypothesize that the three jet solution and two jet solution coexist in the DNS but that the latter has a faster growth rate and lower saturation amplitude than the former.
In order to explore this idea, we biased the CE2 simulation by initializing $c_s$ to
\begin{equation}
  \label{eq:bias}
  c_s(t=0, y_0) = A_0 \lambda cos(2\pi/L_y y_0) + (1-\lambda) cos (\pi/L_y y_0).
\end{equation}
The first term is odd about the center of the domain, while the second is even\footnote{note that both have even parity with respect to the boundary connditions, as required}.
If the linear growth rate explanation is correct, then one would expect that for $\lambda \ll 1$ would lead the CE2 results to pick up the 2-jet solutions.
They do not.
However, CE2 \emph{is} capable of sustaining a 2-jet solution:
by biasing our simulation with $\lambda = 1$ and giving a substantial initial amplitude $A_0 = 40$, the CE2 results do reveal the 2-jet solution, and they saturate at an amplitude comparible to the DNS results.
This strongly suggests that the transition from 3 to 2 jets is the product of a subcritical, non-linear transition mediated by eddy-eddy $\to$ eddy interactions excluded from CE2 (and the previous quasilinear results).



CE2 for bursting?
What does CE2 get for 6 jet?

\subsection{Biasing higher jet counts}
\label{sec:higher_jet}

Try to bias CE2 6 jet parameters.

Start from CE2 2 jet state @ 6 jet parameters?

\subsection{PDFs}
\label{sec:pdf}

QL pdf vs NL pdf.
Does this do better than RB convection?

\section{To Do}
\label{sec:todo}

\begin{todolist}
  \item Fix drag runs
    \begin{todolist}
      \item[\done] restart a working run, adding small $C$. Does it crash?
      \item run with each friction term off, one at a time
    \end{todolist}
  \item Look at 2nd cumulants
  \item DNS with subtracted means: movie. Try to see if ``satellite modes'' are important.
  \item plot DNS $k_x$ at a given height vs time.
  \item Understand why QL/CE2 fail to reproduce correct number of jets. Is it that they can find all solutions, but cannot transition between them?
\end{todolist}

\section{Discussion}
\label{sec:discussion}

Emerging story: \emph{zonal} QL/CE2 fail to transition among states because they preclude $k_x = 1$ ``satellite'' modes. If we can restore the transition between states by going to \emph{ensemble} CE2, then this would give good evidence that that is the right approach.


We acknowledge people for things and grants for money. 

\bibliographystyle{jfm}
% Note the spaces between the initials
\bibliography{busse}

\end{document}

%  LocalWords:  DNS cumulants cumulant annulus
