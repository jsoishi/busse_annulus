
\documentclass{paper}
\usepackage{physics}
\usepackage{hyperref}
\newcommand*{\reynolds}[1]{\mean{#1} + #1'}
\newcommand*{\mean}[1]{\overline{#1}}
\newcommand*{\zonal}[1]{\frac{1}{L_x} \int_0^{L_x} #1 \dd x_1}
\newcommand*{\qderiv}[3][]{\pdv{#2}{x_{#1}}\pdv{#3}{y_{#1}}}
\newcommand{\cz}{c_\zeta}
\newcommand{\czz}{c_{\zeta\zeta}}
\newcommand{\czzz}{c_{\zeta\zeta\zeta}}

\newcommand{\czp}{c_{\zeta\psi}}
\newcommand{\cpz}{c_{\psi\zeta}}

% position vectors: \rr1 will give r_1, \rr2 gives r_2, etc.
\newcommand{\rr}[1]{\vb{r_{#1}}}

% expanding Jacobians. Optional first argument gives subscript for derivatives.
\newcommand{\Jac}[3][]{\pdv{#2}{x_{#1}}\pdv{#3}{y_{#1}} - \pdv{#2}{y_{#1}}\pdv{#3}{x_{#1}}}
\title{How to DSS}

\begin{document}
\maketitle

\section{Preliminaries}
\label{sec:preliminaries}

Let's start with the canonical DSS example, the formation of jets on a stochastically forced, 2D $\beta$ plane.

The equations of motion are

\begin{equation}
  \label{eq:zeta_eom}
  \pdv{\zeta}{t} + J(\psi, \zeta) + \beta \pdv{\psi}{x} = -\kappa \zeta + \nu \laplacian{\zeta} + \eta
\end{equation}
%
\begin{equation}
  \label{eq:zeta_def}
  \zeta = \laplacian{\psi},
\end{equation}
where $\eta$ is some kind of forcing function. The first step is to perform a Reynolds decomposition, $\zeta = \mean{\zeta} + \zeta'$ and $\psi = \mean{\psi} + \psi'$.

\subsection{Averages and Reynolds Rules}
\label{sec:averages}

There are lots of averages one can use, but we will simply use the \emph{zonal average}. Regardless, the only thing we require is that the averaging proceedure respect the Reynolds Rules. A really good exposition on them is \href{https://kiwi.atmos.colostate.edu/group/dave/pdf/Reynolds_Averaging.pdf}{here}. For now, let's just write down the most important ones:
\begin{equation}
  \label{eq:reyn_1}
  \mean{\mean{f} g} = \mean{f}\mean{g}
\end{equation}
\begin{equation}
  \label{eq:reyn_2}
  \mean{f'} = 0
\end{equation}
\begin{equation}
  \label{eq:reyn_3}
  \mean{f + g} = \mean{f} + \mean{g}
\end{equation}

There are two important types of averages we will use. The simplest conceptually, and perhaps the most ``realistic'' is the \emph{ensemble} average,
\begin{equation}
  \label{eq:ensemble_average}
  \mean{f(\rr1)} = \frac{1}{N} \sum_i f_i(\rr1),\  i \in \{1 \ldots N\},
\end{equation}
where $f_i(\rr1)$ is the quantity on the $i$-th trial. We assume the experiment has been performed $N$ times, and we average over all trials. This means that $\mean{f}$ has the same dimensions as each $f_i$. In practice, this is time consuming and difficult. For DSS, it has the problem that the dimensionality grows quickly, as we will see.

The second important average is the \emph{zonal} average, in which we assume one dimension (conventionally $x$) is periodic, that is, it's a representation of the azimuthal angle on a sphere at a given latitude,
\begin{equation}
  \label{eq:zonal_average}
  \mean{f}(y_1) = \zonal{f(x_1, y_1)}. 
\end{equation}
Note that now $\mean{f}$ has lower dimension than $f(\rr1)$ did. It is only a function of $y_1$, since we integrated over $x_1$.

\subsection{Dirac Delta Functions}
\label{sec:dirac-delta-funct}

[Insert usual caution about Dirac deltas not really being functions.]

We will make extensive use of two property of Dirac delta functions. First, the defining property of the derivative of a delta function is
\begin{equation}
  \label{eq:dirac_derivative}
  \int f(x) \dv{\delta(x)}{x} \dd{x} = -\int \dv{f(x)}{x} \delta(x) \dd{x}.
\end{equation}

As a consequence of this,
\begin{equation}
  \label{eq:dirac_deriv_swap}
  \pdv{\delta(y_1 - y_2)}{y_1} = -\pdv{\delta(y_1 - y_2)}{y_2}.
\end{equation}

\section{First Cumulant Equation of Motion}
\label{sec:first-cumul-equat}

We substitute our Reynolds decomposition into equation~(\ref{eq:zeta_eom}),
\begin{equation}
  \label{eq:zeta_eom_expanded}
  \pdv{(\reynolds{\zeta})}{t} + J(\reynolds{\psi}, \reynolds{\zeta}) + \beta \pdv{(\reynolds{\psi})}{x} = -\kappa (\reynolds{\zeta}) + \nu \laplacian{(\reynolds{\zeta})} + \eta,
\end{equation}
and then take the average of the entirety of equation~(\ref{eq:zeta_eom_expanded}). This gives us an equation for the average, which is also the \emph{first cumulant}. The important thing to note is what happens to the non-linear term, $J(\reynolds{\psi}, \reynolds{\zeta})$, when we average it. You can convince yourself quite quickly that $J(\reynolds{\psi}, \reynolds{\zeta}) = J(\mean{\psi}, \mean{\zeta}) + J(\psi', \zeta') + J(\psi', \mean{\zeta}) + J(\mean{\psi}, \zeta')$. By Reynolds rules, the seond two are both zero under averaging.

Furthermore, $\mean{J(\mean{\psi}, \mean{\zeta})} = 0$, if we are using zonal averaging, because by definition of a zonal average, $\pdv*{\mean{f}}{x} = 0$. This leaves only $\mean{J(\psi', \zeta')}$,
% 
\begin{equation}
  \label{eq:zeta_first_c}
  \pdv{\mean{\zeta}}{t} + \mean{J(\psi', \zeta')}  = -\kappa (\mean{\zeta}) + \nu \laplacian{(\mean{\zeta})},
\end{equation}
because $\mean{\eta} = 0$ by definition of the forcing.

Our goal is to derive a set of equations for the cumulants $\cz$, $\czz$, $\ldots$ solely in terms of one another. The first cumulant $\cz = \mean{\zeta}$, so our first task is to get $\mean{J(\psi', \zeta')}$ in terms of other cumulants.

The second cumulant $\czz = \mean{\zeta' \zeta'} = \mean{\zeta'(x_1, y_1)} \mean{\zeta'(x_2, y_2)}$ defines a two-point correlation function of the \emph{fluctuations}. Be careful about the notation; the subscripts on $c$ are not written with the $'$! Because there are two sets of coordinates, we need to be careful about our derivatives. We can define $J_1$ and $J_2$ to mean Jacobians with derivatives on the first or second set of coordinates, respectively.

Turning back to equation~(\ref{eq:zeta_first_c}), we note that this term involves the two variables $\zeta'(x_1, y_1)$ and $\psi'(x_1, y_1)$ \emph{at the same point!} For convenience, let's define position vectors $\rr1 = (x_1, y_1)$ and similarly for $\rr2$. In order to transform this into something involving $\czz$ and $\czp$, we need to first move one variable to the other coordinate. We do this by invoking Dirac $\delta$ functions,

\begin{equation}
  \label{eq:var_change}
  \psi'(\rr1) = \int \psi'(\rr2) \delta(\rr1 - \rr2) \dd{\rr2}.
\end{equation}
It's important to note that $\delta(x)$ is symmetric about zero, so equation~(\ref{eq:var_change}) is the same when you exchange $\rr1$ and $\rr2$.

\textbf{CHANGE THIS TO REFLECT SIGNS IN 2013 PRL: make $\zeta'$, not $\psi'$ switch to $\rr2$...}

So, let's now explicitly include spatial dependencies,
\begin{equation}
  \label{eq:expanded_J1}
  \mean{J_1(\zeta'(\rr1), \psi'(\rr1))} = \Jac[1]{\zeta'(\rr1)}{\psi'(\rr1)}.
\end{equation}
Now, we substitute $\psi'(\rr1)$ using equation~(\ref{eq:var_change}),
\begin{equation}
  \label{eq:expanded_J1_int}
  = \mean{\Jac[1]{\zeta'(\rr1)}{\int \psi'(\rr2) \delta(\rr1 - \rr2) \dd \rr2}}.
\end{equation}
Next, we swap the order of the integral,
\begin{equation}
  \label{eq:expanded_J1_int_swap}
  = \mean{\int{\Jac[1]{\zeta'(\rr1)}{(\psi'(\rr2) \delta(\rr1 - \rr2)) }\dd \rr2}},
\end{equation}
and notice that $\psi'(\rr2)$ does not depend on $\rr1$, so it can be pulled out of second argument of the Jacobian and put in the first one.
\begin{equation}
  \label{eq:expanded_J1_int_combined}
  \begin{split}
  & = \mean{\int{\Jac[1]{(\zeta'(\rr1)\psi'(\rr2))}{\delta(\rr1 - \rr2) }\dd \rr2}}\\
  & = \mean{J_1(\zeta'(\rr1)\psi'(\rr2),\delta(\rr1 - \rr2))}.
\end{split}
\end{equation}
We're almost there, but we have the mean of the Jacobian, and we want the Jacobian of the mean of $\zeta'(\rr1)\psi'(\rr2)$ (which is $\czp$).


The fact that we can swap means and derivatives is obvious for ensemble averages (since only $\zeta'$ and $\psi'$ depend on the trial number, but $\delta(\rr1 - \rr2)$ does not). For zonal averages it is much trickier, because now we're integrating over $x_1$ and $\delta(\rr1-\rr2) = \delta(x_1 - x_2) \delta(y_1 - y_2)$ clearly depends on $x_1$.

\textbf{The following argument is very ugly, and I'm not 100\% sure it's right.}
Zonal averaging of two quantities is a bit tricky, because we want to preserve non-locality. This means that we \emph{don't} do
\begin{equation}
  \label{eq:zonal-wrong}
  \mean{fg}(y) = \frac{1}{L_x} \int_0^{L_x} f(x, y) g(x, y) \dd{x},
\end{equation}
but instead do
\begin{equation}
  \label{eq:zonal-right}
  \mean{fg}(\xi, y_1, y_2) = \frac{1}{L_x} \int_0^{L_x} f(x_1, y_1) g(x_1 - \xi, y_2) \dd{x_1},
\end{equation}
where $\xi = x_1 - x_2$ is the difference coordinate in $x$. Be careful! It's very easy to think of keeping complete non-locality, that is, $f(x_1, y_1)$ and $g(x_2, y_2)$ when doing a zonal average, but that doesn't quite make sense:
\begin{equation}
  \label{eq:zonal-not-right}
  \mean{fg}(x_2, y_1, y_2) = \frac{1}{L_x} \int_0^{L_x} f(x_1, y_1) g(x_2, y_2) \dd{x_1},
\end{equation}
becuase now you could just pull $g(x_2, y_2)$ out of the integral since it doesn't depend on $x_1$! This is clearly not what we want.

With that in mind, we can clearly define $\czp$,
\begin{equation}
  \label{eq:czp_def}
  \czp(\xi, y_1, y_2) = \zonal{\zeta'(x_1, y_1) \psi'(x_1 - \xi, y_2)}.
\end{equation}
Given this, let's pick up from equation~(\ref{eq:expanded_J1_int_combined}), using $\xi$ strategically,
\begin{equation}
  \label{eq:}
  \mean{\int{\Jac[1]{(\zeta'(x_1, y_1)\psi'(x_1- \xi))}{\delta(\xi, y_1- y_2) }\dd \rr2}},
\end{equation}
we can now commit to the zonal average,
\begin{equation}
  \label{eq:}
  \zonal{\int{\Jac[1]{(\zeta'(x_1, y_1)\psi'(x_1- \xi))}{\delta(\xi, y_1- y_2) }\dd \rr2}},
\end{equation}
and after swapping the order of integration, we can easily see that the term involving $\delta(\xi, y_1 - y_2)$ can be pulled out of the integral over $x_1$. Focusing on the second term, we must make a sly substitution: $\pdv*{\delta(\xi, y_1 - y_2)}{x_1} = \pdv*{\delta(\xi, y_1 - y_2)}{\xi}$, and this is in turn independent of $x_1$.  Thus we can write
\begin{equation}
  \label{eq:zonal_hack}
  \begin{split}
     \frac{1}{L_x} \int \dd{\rr2} & \left[\pdv{\delta(\xi, y_1 - y_2)}{y_1} \int_0^{L_x} \pdv{\zeta'(x_1, y_1)\psi'(x_1- \xi)}{x_1} \dd{x_1} \right.\\
      - &\left. \pdv{\delta(\xi, y_1 - y_2)}{\xi} \int_0^{L_x} \pdv{\zeta'(x_1, y_1)\psi'(x_1- \xi)}{y_1} \dd{x_1} \right].    
  \end{split}
\end{equation}
Changing $\pdv*{\delta(\xi, y_1 - y_2)}{\xi}$ back into a derivative in $x_1$ and invoking the Reynolds rule that says $\mean{\pdv*{f}{x}} = \pdv*{\mean{f}}{x}$, we arrive at our final result,
\begin{equation}
  \label{eq:zonal_hack_2}
  \begin{split}
    \int \dd{\rr2} & \left[\pdv{\delta(\xi, y_1 - y_2)}{y_1} \pdv{\mean{\zeta'(x_1, y_1)\psi'(x_1- \xi)}}{x_1} \right.\\
    - & \left.\pdv{\delta(\xi, y_1 - y_2)}{x_1} \pdv{\mean{\zeta'(x_1, y_1)\psi'(x_1- \xi)}}{y_1} \dd{x_1}\right]\\
     = &\int \dd{\rr2} J_1(\czp, \delta(\rr1 - \rr2)).
  \end{split}
\end{equation}

\textbf{From here, I've assumed we have $\zeta'$ at $\rr2$.}

The next step is rather simple. Given the definition of $\xi$, we can construct differentials $\dd{\xi} = -\dd{x_2} = \dd{x_1}$, and we'll deploy them in expanding the integral over $\rr2$. Noting that $\dd{\rr2} = \dd{x_2}\dd{y_2}$,
\begin{equation}
  \label{eq:expand_int_rr2}
  - \int \dd{\xi} \dd{y_2} \left[ \delta(y_1 - y_2)\pdv{\delta(\xi)}{\xi} \pdv{\cpz}{y_1} - \pdv{\cpz}{\xi} \delta(\xi)\pdv{\delta(y_1 - y_2)}{y_1}\right].
\end{equation}
We'll use a few properties of Dirac deltas (see section~\ref{sec:dirac-delta-funct}). Combining these with equation~(\ref{eq:expand_int_rr2}),
\begin{equation}
  \label{eq:expand_int_rr2_dirac}
  \begin{split}
    & - \int \dd{\xi} \dd{y_2} \left[ \delta(\xi)\delta(y_1 - y_2)\pdv{\cpz}{\xi}{y_1} + \delta(\xi)\delta(y_1 - y_2)\pdv{\cpz}{\xi}{y_2}\right]\\
    = & -\eval{\qty(\pdv{y_1} + \pdv{y_2}) \pdv{\cpz}{\xi}}_{y_1 \to y_2}^{\xi \to 0},
  \end{split}
\end{equation}
where the $\cz \to 0$ and $y_1 \to y_2$ are simply the consequences of integration against the delta functions.

Finally, we have the mean Jacobian in terms of a higher order moment,
\begin{equation}
  \label{eq:first_cumulant_EOM}
  \pdv{\cz}{t} = -\eval{\qty(\pdv{y_1} + \pdv{y_2}) \pdv{\cpz}{\xi}}_{y_1 \to y_2}^{\xi \to 0} - \kappa \cz + \nu \laplacian{\cz}.
\end{equation}
The most important part of this equation is the fact that non-trivial $\cpz$ can come into equilibrium with the dissipation and friction terms, leading to non-zero mean vorticity.

\section{Second Cumulant Equation of Motion}
\label{sec:second-cumul-equat}
At the end of the previous section, we derived an equation for the first cumulant $\cz$ in terms of the second cumulant, $\cpz$. We now need an equation for this second cumulant. We expect that this equation will contain yet a higher order cumulant, and we will eventually need to find some way of ``closing'' the set of equations: finding a way to stop the increase of order.

What we need is $\cpz$ for equation~(\ref{eq:first_cumulant_EOM}), but it is much more straight forward to construct an eqeuation for $\czz$. Luckily, because $\cpz = \laplacian_1{\czz}$, this isn't a problem.

Returing to equation~(\ref{eq:zeta_eom_expanded}), we want to manipulate it into an equation for $\czz$. To do this, we first invoke the chain rule,
\begin{equation}
  \label{eq:chain_rule}
  \pdv*{\mean{\zeta' \zeta'}}{t} = 2\zeta'\pdv*{\zeta'}{t}
\end{equation}


Now, we take equation~(\ref{eq:zeta_eom_expanded}), move the first cumulant time derivative to the right hand side, multiply by $\zeta'$ and average,
\begin{equation}
  \label{eq:2c_eom_start}
  \mean{\zeta' \pdv{\zeta'}{t}} = \mean{-\zeta' \pdv{\mean{\zeta}}{t} - \zeta' J(\reynolds{\psi}, \reynolds{\zeta}) - \zeta' \beta \pdv{\reynolds{\psi}}{x} - \kappa \zeta' \qty(\reynolds{\zeta}) + \kappa \zeta' \laplacian{\qty(\reynolds{\zeta})} + \zeta' \eta}.
\end{equation}
All terms proportional to a fluctuation times a mean vanish under another mean, so we drop the first, third, fifth, and seventh terms (counting the Jacobian as a single term for now).

Using equation~(\ref{eq:chain_rule}), we arrive at
\begin{equation}
  \label{eq:2c_eom_jacobian}
  \pdv{\mean{\zeta' \zeta'}}{t} = 2 \qty[\mean{-\zeta' J(\reynolds{\psi}, \reynolds{\zeta})} - \beta \mean{\zeta' \pdv{\psi'}{x}} - \kappa \mean{\zeta' \zeta'} + \nu \mean{\zeta' \laplacian{\zeta'}} + \mean{\zeta' \eta}].
\end{equation}
There are two main operations we need to do: delocalization and symmetrization. Delocalization refers to the process of expanding part of each quadratic term into $\rr2$. By delocalizing just the time derivative term, we can see what it will do, and what form it forces the rest of the terms to take. Using equation~(\ref{eq:var_change}),

\begin{equation}
  \label{eq:deloaclized_dt}
  \begin{split}
    \pdv{\mean{\zeta'(\rr1) \zeta'(\rr1)}}{t} &= \pdv{t} \qty(\mean{\zeta'(\rr1) \int \zeta'(\rr2) \delta(\rr1 - \rr2) \dd{\rr2}})\\
    & = -\pdv{t} \qty(\mean{\int \zeta'(x_1, y_1) \zeta'(x_1 - \xi, y_2) \delta(\xi) \delta(y_1 - y_2) \dd{\xi} \dd{y_2}})\\
    & = -\int \pdv{\czz}{t} \delta(\xi) \delta(y_1 - y_2) \dd{\xi} \dd{y_2}.
  \end{split}
\end{equation}
The key point to the last step is that under a zonal average, we can swap order of integration and remove the delta functions from the inner averaging integral. Thus, we need all terms on the right hand side to be of the form
\begin{equation}
  \label{eq:2c_form}
  \int Q \delta(\xi) \delta(y_1 - y_2) \dd{\xi} \dd{y_2},
\end{equation}
where $Q$ is a given zonally averaged quantity.

Let's turn to the Jacobian. We assume zonal averaging, so we can directly eliminate $\pdv*{\mean{\zeta}}{x}$ and $\pdv*{\mean{\psi}}{x}$. This leaves
\begin{equation}
  \label{eq:2c_jacobian}
  \begin{split}
    \mean{\zeta' J(\reynolds{\zeta}, \reynolds{\psi})} &= \mean{\zeta'\qty(\qderiv{\zeta'}{\mean{\psi}} + \qderiv{\zeta'}{\psi'} - \qderiv{\psi'}{\mean{\zeta}} - \qderiv{\psi'}{\zeta'})}\\
    &= \mean{\zeta'\qty(\qderiv{\zeta'}{\mean{\psi}} - \qderiv{\psi'}{\mean{\zeta}})}\\
  \end{split}
\end{equation}
We note that the second and fourth terms are \emph{cubic} in perturbations, and thus will be dropped in a CE2\footnote{it might be nice to put these in terms of higher cumulants, to fully drive home the underlying expansion}. Now taking the first term, delocalizing one of the $\zeta'$, and noting that the $\mean{\psi}$ term can be pulled out of the mean by Reynolds rules, we get
\begin{equation}
  \label{eq:2c_delocalize}
  \begin{split}
    \mean{\zeta' \qderiv{\zeta'}{\mean{\psi}}} &= \mean{\zeta'(\rr1) \pdv{x_1} \qty(\int \zeta'(\rr2) \delta(\rr1 - \rr2) \dd{\rr2})} \pdv{\mean{\psi}}{y_1}\\
    &= -\mean{\int \zeta'(x_1, y_1) \pdv{\xi} \qty(\zeta'(x_1 - \xi, y_2)  \delta(\xi)\delta(y_1 - y_2)) \dd{\xi} \dd{y_2}} \pdv{\mean{\psi}}{y_1}\\
    &= -\mean{\int \pdv{\xi} \qty(\zeta'(x_1, y_1) \zeta'(x_1 - \xi, y_2)  \delta(\xi)) \delta(y_1 - y_2) \dd{\xi} \dd{y_2}} \pdv{\mean{\psi}}{y_1}
  \end{split}
\end{equation}
liwhere in the second step, we transformed to $\xi, y_1, y_2$ and the third simply rearranges terms. Now, finally, we apply zonal averaging, swap the order of integration, and note that only $\zeta'(x_1, y_1) \zeta'(x_1 - \xi, y_2)$ depend on $x_1$. This allows us to write
\begin{equation}
  \label{eq:2c_delocalize_final}
  \begin{split}
    \mean{\zeta' \qderiv{\zeta'}{\mean{\psi}}} &= -\int \pdv{\xi}\qty(\czz \delta(\xi)) \delta(y_1 - y_2) \dd{\xi} \dd{y_2} \pdv{\mean{\psi}}{y_1}\\
   & = - 2\int \pdv{\czz}{\xi} \pdv{\mean{\psi}}{y_1} \delta(\xi) \delta(y_1 - y_2) \dd{\xi} \dd{y_2} 
  \end{split}
\end{equation}
where in the last step we used the chain rule, the derivative property of the delta function, and moved the $\mean{\psi}$ term into the integral. The term is now clearly in the form given by equation~(\ref{eq:2c_form}). The other remaining term in the Jacobian is similar.

Next, we turn to the $\beta$ term,
\begin{equation}
  \begin{split}
    -\beta \mean{\zeta' \pdv{\psi'}{x}} &= - \beta \mean{\int \zeta'(x_2, y_2) \dd{\rr2} \pdv{\psi'(x_1, y_1)}{x_1}}\\
    &= \beta \mean{\int \zeta'(x_1 - \xi, y_2) \pdv{\psi'(x_1, y_1)}{x_1} \dd{\xi} \dd{y_2}}\\
    &= \beta \int \pdv{\cpz}{\xi} \dd{\xi} \dd{y_2}
    \end{split}
  \label{eq:2c_beta_term}
\end{equation}
If we swap $\rr1$ and $\rr2$, we have
\begin{equation}
  \begin{split}
    -\beta \mean{\zeta' \pdv{\psi'}{x}} &= - \beta \mean{\zeta'(x_1, y_1) \pdv{x_2} \int  \psi'(x_2, y_2)\dd{\rr2}}\\
    &= \beta \mean{\zeta'(x_1, y_1) \int \pdv{\psi'(x_2, y_2)}{\xi} \dd{\rr2}}\\
    &= -\beta \mean{\int \pdv{\zeta'(x_1, y_1) \psi'(x_1- \xi, y_2)}{\xi} \dd{\xi} \dd{y_2}}\\
    &= -\beta \int \pdv{\czp}{\xi} \dd{\xi} \dd{y_2}
    \end{split}
  \label{eq:2c_beta_term_swap}
\end{equation}
Combining these terms, the $\beta$ term is
\begin{equation}
  \label{eq:2c_final_beta}
  \boxed{-\beta \mean{\zeta' \pdv{\psi'}{x}} = \frac{\beta}{2} \qty(\int \pdv{\cpz}{\xi} -\pdv{\czp}{\xi}  \dd{\xi}\dd{y_2})}
\end{equation}
Here, things are a bit more complicated in all terms, not just the Jacobian, because we must symmetrize all terms that are quadratic in fluctuations. ``Symmetrization'' here refers to making sure that the overall terms are symmetric under swapping of $\rr1$ and $\rr2$, which we do in a simple way, using curly braces to denote the operator
\begin{equation}
  \label{eq:symmetrization}
  \qty{c(\rr1, \rr2)} = \frac{1}{2} \qty(c(\rr1, \rr2) + c(\rr2, \rr1)).
\end{equation}

\section{Cumulants and Centered Moments}
\label{sec:cumul-cent-moments}

Here, put the results from pages 2-4 of handwritten notes dated 18 Apr 2018.
\end{document}

