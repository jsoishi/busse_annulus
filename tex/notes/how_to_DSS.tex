
\documentclass{paper}
\usepackage{physics}
\usepackage{hyperref}
\newcommand*{\reynolds}[1]{\mean{#1} + #1'}
\newcommand*{\mean}[1]{\overline{#1}}
\newcommand*{\zonal}[1]{\frac{1}{L_x} \int_0^{L_x} #1 \dd x_1}
\newcommand{\cz}{c_\zeta}
\newcommand{\czz}{c_{\zeta\zeta}}
\newcommand{\czzz}{c_{\zeta\zeta\zeta}}

\newcommand{\czp}{c_{\zeta\psi}}
\newcommand{\cpz}{c_{\psi\zeta}}

% position vectors: \rr1 will give r_1, \rr2 gives r_2, etc.
\newcommand{\rr}[1]{\vb{r_{#1}}}

% expanding Jacobians. Optional first argument gives subscript for derivatives.
\newcommand{\Jac}[3][]{\pdv{#2}{x_{#1}}\pdv{#3}{y_{#1}} - \pdv{#2}{y_{#1}}\pdv{#3}{x_{#1}}}
\title{How to DSS}

\begin{document}
\maketitle

\section{Preliminaries}
\label{sec:preliminaries}

Let's start with the canonical DSS example, the formation of jets on a stochastically forced, 2D $\beta$ plane.

The equations of motion are

\begin{equation}
  \label{eq:zeta_eom}
  \pdv{\zeta}{t} + J(\psi, \zeta) + \beta \pdv{\psi}{x} = -\kappa \zeta + \nu \laplacian{\zeta} + \eta
\end{equation}
%
\begin{equation}
  \label{eq:zeta_def}
  \zeta = \laplacian{\psi},
\end{equation}
where $\eta$ is some kind of forcing function. The first step is to perform a Reynolds decomposition, $\zeta = \mean{\zeta} + \zeta'$ and $\psi = \mean{\psi} + \psi'$.

\subsection{Averages and Reynolds Rules}
\label{sec:averages}

There are lots of averages one can use, but we will simply use the \emph{zonal average}. Regardless, the only thing we require is that the averaging proceedure respect the Reynolds Rules. A really good exposition on them is \href{https://kiwi.atmos.colostate.edu/group/dave/pdf/Reynolds_Averaging.pdf}{here}. For now, let's just write down the most important ones:
\begin{equation}
  \label{eq:reyn_1}
  \mean{\mean{f} g} = \mean{f}\mean{g}
\end{equation}
\begin{equation}
  \label{eq:reyn_2}
  \mean{f'} = 0
\end{equation}
\begin{equation}
  \label{eq:reyn_3}
  \mean{f + g} = \mean{f} + \mean{g}
\end{equation}

There are two important types of averages we will use. The simplest conceptually, and perhaps the most ``realistic'' is the \emph{ensemble} average,
\begin{equation}
  \label{eq:ensemble_average}
  \mean{f(\rr1)} = \frac{1}{N} \sum_i f_i(\rr1),\  i \in \{1 \ldots N\},
\end{equation}
where $f_i(\rr1)$ is the quantity on the $i$-th trial. We assume the experiment has been performed $N$ times, and we average over all trials. This means that $\mean{f}$ has the same dimensions as each $f_i$. In practice, this is time consuming and difficult. For DSS, it has the problem that the dimensionality grows quickly, as we will see.

The second important average is the \emph{zonal} average, in which we assume one dimension (conventionally $x$) is periodic, that is, it's a representation of the azimuthal angle on a sphere at a given latitude,
\begin{equation}
  \label{eq:zonal_average}
  \mean{f}(y_1) = \zonal{f(x_1, y_1)}. 
\end{equation}
Note that now $\mean{f}$ has lower dimension than $f(\rr1)$ did. It is only a function of $y_1$, since we integrated over $x_1$.

\section{First Cumulant Equation of Motion}
\label{sec:first-cumul-equat}

We substitute our Reynolds decomposition into equation~(\ref{eq:zeta_eom}),
\begin{equation}
  \label{eq:zeta_eom_expanded}
  \pdv{(\reynolds{\zeta})}{t} + J(\reynolds{\psi}, \reynolds{\zeta}) + \beta \pdv{(\reynolds{\psi})}{x} = -\kappa (\reynolds{\zeta}) + \nu \laplacian{(\reynolds{\zeta})} + \eta,
\end{equation}
and then take the average of the entirety of equation~(\ref{eq:zeta_eom_expanded}). This gives us an equation for the average, which is also the \emph{first cumulant}. The important thing to note is what happens to the non-linear term, $J(\reynolds{\psi}, \reynolds{\zeta})$, when we average it. You can convince yourself quite quickly that $J(\reynolds{\psi}, \reynolds{\zeta}) = J(\mean{\psi}, \mean{\zeta}) + J(\psi', \zeta') + J(\psi', \mean{\zeta}) + J(\mean{\psi}, \zeta')$. By Reynolds rules, the seond two are both zero under averaging.

Furthermore, $\mean{J(\mean{\psi}, \mean{\zeta})} = 0$, if we are using zonal averaging, because by definition of a zonal average, $\pdv*{\mean{f}}{x} = 0$. This leaves only $\mean{J(\psi', \zeta')}$,
% 
\begin{equation}
  \label{eq:zeta_first_c}
  \pdv{\mean{\zeta}}{t} + \mean{J(\psi', \zeta')}  = -\kappa (\mean{\zeta}) + \nu \laplacian{(\mean{\zeta})},
\end{equation}
because $\mean{\eta} = 0$ by definition of the forcing.

Our goal is to derive a set of equations for the cumulants $\cz$, $\czz$, $\ldots$ solely in terms of one another. The first cumulant $\cz = \mean{\zeta}$, so our first task is to get $\mean{J(\psi', \zeta')}$ in terms of other cumulants.

The second cumulant $\czz = \mean{\zeta' \zeta'} = \mean{\zeta'(x_1, y_1)} \mean{\zeta'(x_2, y_2)}$ defines a two-point correlation function of the \emph{fluctuations}. Be careful about the notation; the subscripts on $c$ are not written with the $'$! Because there are two sets of coordinates, we need to be careful about our derivatives. We can define $J_1$ and $J_2$ to mean Jacobians with derivatives on the first or second set of coordinates, respectively.

Turning back to equation~(\ref{eq:zeta_first_c}), we note that this term involves the two variables $\zeta'(x_1, y_1)$ and $\psi(x_1, y_1)$ \emph{at the same point!} For convenience, let's define position vectors $\rr1 = (x_1, y_1)$ and similarly for $\rr2$. In order to transform this into something involving $\czz$ and $\czp$, we need to first move one variable to the other coordinate. We do this by invoking Dirac $\delta$ functions,

\begin{equation}
  \label{eq:var_change}
  \psi'(\rr2) = \int \psi'(\rr1) \delta(\rr1 - \rr2) \dd{\rr1}.
\end{equation}
It's important to note that $\delta(x)$ is symmetric about zero, so equation~(\ref{eq:var_change}) is the same when you exchange $\rr1$ and $\rr2$.

\textbf{CHANGE THIS TO REFLECT SIGNS IN 2013 PRL: make $\zeta'$, not $\phi'$ switch to $\rr2$...}

So, let's now explicitly include spatial dependencies,
\begin{equation}
  \label{eq:expanded_J1}
  \mean{J_1(\zeta'(\rr1), \psi'(\rr1))} = \Jac[1]{\zeta'(\rr1)}{\psi'(\rr1)}.
\end{equation}
Now, we substitute $\psi'(\rr1)$ using equation~(\ref{eq:var_change}),
\begin{equation}
  \label{eq:expanded_J1_int}
  = \mean{\Jac[1]{\zeta'(\rr1)}{\int \psi'(\rr2) \delta(\rr1 - \rr2) \dd \rr2}}.
\end{equation}
Next, we swap the order of the integral,
\begin{equation}
  \label{eq:expanded_J1_int_swap}
  = \mean{\int{\Jac[1]{\zeta'(\rr1)}{(\psi'(\rr2) \delta(\rr1 - \rr2)) }\dd \rr2}},
\end{equation}
and notice that $\psi'(\rr2)$ does not depend on $\rr1$, so it can be pulled out of second argument of the Jacobian and put in the first one.
\begin{equation}
  \label{eq:expanded_J1_int_combined}
  \begin{split}
  & = \mean{\int{\Jac[1]{(\zeta'(\rr1)\psi'(\rr2))}{\delta(\rr1 - \rr2) }\dd \rr2}}\\
  & = \mean{J_1(\zeta'(\rr1)\psi'(\rr2),\delta(\rr1 - \rr2))}.
\end{split}
\end{equation}
We're almost there, but we have the mean of the Jacobian, and we want the Jacobian of the mean of $\zeta'(\rr1)\psi'(\rr2)$ (which is $\czp$).


The fact that we can swap means and derivatives is obvious for ensemble averages (since only $\zeta'$ and $\psi'$ depend on the trial number, but $\delta(\rr1 - \rr2)$ does not). For zonal averages it is much trickier, because now we're integrating over $x_1$ and $\delta(\rr1-\rr2) = \delta(x_1 - x_2) \delta(y_1 - y_2)$ clearly depends on $x_1$.

\textbf{The following argument is very ugly, and I'm not 100\% sure it's right.}
Zonal averaging of two quantities is a bit tricky, because we want to preserve non-locality. This means that we \emph{don't} do
\begin{equation}
  \label{eq:zonal-wrong}
  \mean{fg}(y) = \frac{1}{L_x} \int_0^{L_x} f(x, y) g(x, y) \dd{x},
\end{equation}
but instead do
\begin{equation}
  \label{eq:zonal-right}
  \mean{fg}(\xi, y_1, y_2) = \frac{1}{L_x} \int_0^{L_x} f(x_1, y_1) g(x_1 - \xi, y_2) \dd{x_1},
\end{equation}
where $\xi = x_1 - x_2$ is the difference coordinate in $x$. Be careful! It's very easy to think of keeping complete non-locality, that is, $f(x_1, y_1)$ and $g(x_2, y_2)$ when doing a zonal average, but that doesn't quite make sense:
\begin{equation}
  \label{eq:zonal-not-right}
  \mean{fg}(x_2, y_1, y_2) = \frac{1}{L_x} \int_0^{L_x} f(x_1, y_1) g(x_2, y_2) \dd{x_1},
\end{equation}
becuase now you could just pull $g(x_2, y_2)$ out of the integral since it doesn't depend on $x_1$! This is clearly not what we want.

With that in mind, we can clearly define $\czp$,
\begin{equation}
  \label{eq:czp_def}
  \czp(\xi, y_1, y_2) = \zonal{\zeta'(x_1, y_1) \psi'(x_1 - \xi, y_2)}.
\end{equation}
Given this, let's pick up from equation~(\ref{eq:expanded_J1_int_combined}), using $\xi$ strategically,
\begin{equation}
  \label{eq:}
  \mean{\int{\Jac[1]{(\zeta'(x_1, y_1)\psi'(x_1- \xi))}{\delta(\xi, y_1- y_2) }\dd \rr2}},
\end{equation}
we can now commit to the zonal average,
\begin{equation}
  \label{eq:}
  \zonal{\int{\Jac[1]{(\zeta'(x_1, y_1)\psi'(x_1- \xi))}{\delta(\xi, y_1- y_2) }\dd \rr2}},
\end{equation}
and after swapping the order of integration, we can easily see that the term involving $\delta(\xi, y_1 - y_2)$ can be pulled out of the integral over $x_1$. Focusing on the second term, we must make a sly substitution: $\pdv*{\delta(\xi, y_1 - y_2)}{x_1} = \pdv*{\delta(\xi, y_1 - y_2)}{\xi}$, and this is in turn independent of $x_1$.  Thus we can write
\begin{equation}
  \label{eq:zonal_hack}
  \begin{split}
     \frac{1}{L_x} \int \dd{\rr2} & \left[\pdv{\delta(\xi, y_1 - y_2)}{y_1} \int_0^{L_x} \pdv{\zeta'(x_1, y_1)\psi'(x_1- \xi)}{x_1} \dd{x_1} \right.\\
      - &\left. \pdv{\delta(\xi, y_1 - y_2)}{\xi} \int_0^{L_x} \pdv{\zeta'(x_1, y_1)\psi'(x_1- \xi)}{y_1} \dd{x_1} \right].    
  \end{split}
\end{equation}
Changing $\pdv*{\delta(\xi, y_1 - y_2)}{\xi}$ back into a derivative in $x_1$ and invoking the Reynolds rule that says $\mean{\pdv*{f}{x}} = \pdv*{\mean{f}}{x}$, we arrive at our final result,
\begin{equation}
  \label{eq:zonal_hack_2}
  \begin{split}
    \int \dd{\rr2} & \left[\pdv{\delta(\xi, y_1 - y_2)}{y_1} \pdv{\mean{\zeta'(x_1, y_1)\psi'(x_1- \xi)}}{x_1} \right.\\
    - & \left.\pdv{\delta(\xi, y_1 - y_2)}{x_1} \pdv{\mean{\zeta'(x_1, y_1)\psi'(x_1- \xi)}}{y_1} \dd{x_1}\right]\\
     = &\int \dd{\rr2} J_1(\czp, \delta(\rr1 - \rr2)).
  \end{split}
\end{equation}

\textbf{From here, I've assumed we have $\zeta'$ at $\rr2$.}

The next step is rather simple. Given the definition of $\xi$, we can construct differentials $\dd{\xi} = -\dd{x_2} = \dd{x_1}$, and we'll deploy them in expanding the integral over $\rr2$. Noting that $\dd{\rr2} = \dd{x_2}\dd{y_2}$,
\begin{equation}
  \label{eq:expand_int_rr2}
  - \int \dd{\xi} \dd{y_2} \left[ \delta(y_1 - y_2)\pdv{\delta(\xi)}{\xi} \pdv{\cpz}{y_1} - \pdv{\cpz}{\xi} \delta(\xi)\pdv{\delta(y_1 - y_2)}{y_1}\right].
\end{equation}
We use two property of Dirac delta functions. First, $\pdv*{\delta(y_1 - y_2)}{y_1} = -\pdv*{\delta(y_1 - y_2)}{y_2}$. Second, the defining property of a derivative of a delta function is
\begin{equation}
  \label{eq:dirac_derivative}
  \int f(x) \dv{\delta(x)}{x} \dd{x} = \int \dv{f(x)}{x} \delta(x) \dd{x}.
\end{equation}
Combining these with equation~(\ref{eq:expand_int_rr2}),
\begin{equation}
  \label{eq:expand_int_rr2_dirac}
  \begin{split}
    & - \int \dd{\xi} \dd{y_2} \left[ \delta(\xi)\delta(y_1 - y_2)\pdv{\cpz}{\xi}{y_1} + \delta(\xi)\delta(y_1 - y_2)\pdv{\cpz}{\xi}{y_2}\right]\\
    = & -\eval{\qty(\pdv{y_1} + \pdv{y_2}) \pdv{\cpz}{\xi}}_{y_1 \to y_2}^{\xi \to 0}.    
  \end{split}
\end{equation}
Finally, we have the mean Jacobian in terms of a higher order moment.
\section{Cumulants and Centered Moments}
\label{sec:cumul-cent-moments}

Here, put the results from pages 2-4 of handwritten notes dated 18 Apr 2018.
\end{document}

